\documentclass[conference]{IEEEtran}
\usepackage{url}
\usepackage{graphicx}

\begin{document}

\title{SRS: Venta y distribución de textiles para el sector industrial con enfoque a vestidos de gala}

\author{
\IEEEauthorblockN{
Armenta Telles Jesús Manuel 0321101244,
Contreras Rangel Martin 0322103695, \\
Diaz Escalante José Ángel 0322103701,
Higuera Sánchez Dulce Mariela 0322103734, \\
Reyes Contreras Ramsés 0322103800,
Rodríguez Cacho Ximena Charleene 0322103808}

\IEEEauthorblockA{Grupo: 4-A, Turno: Matutino \\
Tijuana, B.C a 05 de febrero de 2024}
}

\maketitle

\section{INTRODUCCIÓN}

El presente documento tiene por objeto desarrollar una aplicación web que gestione la administación de recursos (materia prima) y maneje una correcta logistica con sus materiales, así como distribuirlos a sus tres sucursales y tener un control de ventas sobre el mismo, para que de esta manera se cuenta con una organización óptima y un manejo eficiente de sus productos, evitanto así tener pérdidas monetarias.

\section{Objetivo}

Optimizar las operaciones comerciales de la empresa a traves de una app web, reduciendo pérdidas en recursos y ofreciendo una experiencia intuitiva para los clientes, facilitando la búsqueda, selección y compra de productos desde un dispositivo movil. 
		
\section{Descripción del proyecto}
Es una aplicación enfocada en administrar las ventas
producidas, le permitirá a la empresa gestionar sus diversos catálogos (de clientes,
de proveedores) así como sus inventarios (de materia y de los lotes de vestidos),
para así lograr venderlos a los clientes, a través de una interfaz intuitiva. \linebreak  

Este proyecto soluciona los problemas de la empresa, con la finalidad de evitar pérdidas
monetarias. El administrador será capaz de actualizar los catálogos e inventarios, así como
agregar nuevos elementos, podrá modificar, eliminar y consultar las existencias del
momento. \linebreak  

Los potenciales clientes podrán solicitar en la aplicación web el tipo de vestido
deseado por lote, y las cantidades que componen el lote vendrá indicado en la
descripción, contarán con su perfil individual, con el cual podrán visualizar en la
interfaz web los productos, agregarlos a un “carrito de compras”, y decidir si
modificara su solicitud, cancelar la solicitud o simplemente consultar sus facturas
realizadas, además de tener acceso a modificar su perfil de usuario.
		
\section{Carácteristicas del sistema}

De forma general, este se enfocará en gestione la administración de recursos, y que coordine eficientemente la logística de distribución hacia las tres sucursales de la empresa. Además, el sistema controlará las ventas, ofreciendo una aplicación intuitiva para usuarios y administradores, permitiendo la gestión de catálogos de productos, clientes y proveedores, así como la realización de pedidos, Y la gestión del carrito de compras.

\section{Funcionalidades generales}

El sistema contará con las siguientes carácteristicas fundamentales para su correcto funcionamiento:

\begin{itemize}
	\item Registro de materia prima y otros recursos.
	\item Gestión de inventario de materiales y productos.
	\item Coordinación logística para la distribución a las sucursales.
	\item Control de ventas, incluyendo catálogos de productos, clientes y proveedores.
	\item Plataforma intuitiva para usuarios y administradores.
	\item Realización de pedidos y gestión de carrito de compras.
	\item Registros de compras.
	\item Actualización de catálogos e inventarios por parte de los administradores.
	\item Funciones de modificación, eliminación y consulta de existencias.
	\item Organización eficiente de productos y recursos de la empresa.
\end{itemize}

\section{Requerimientos de la aplicación}

La aplicación tiene por nombre de software: Galatex, sus requerimientos son:
		
\begin{itemize}
	\item Inventario: 
	\subitem  Se manejará un inventario donde se
	tenga catalogado cada material y
	producto con el que contamos, para así
	facilitar el control y nuestro manejo de
	materiales.
	
	\item Catálogo de productos:
	\subitem Se tendrá un inventario donde se
	muestren todos productos disponibles
	en el momento (los lotes de vestido), y
	en caso de estar agotado igual se
	mostrará en pantalla.
	
	\item Catálogo de proveedores:
	\subitem Se tendrá registrado cada proveedor de
	material, guardando información como
	el nombre de la empresa, ubicación, su
	código de referencia fiscal y el material
	que nos surten.
	\subitem
	\subitem
	
	\item Catálogo de clientes:
	\subitem Así como se tendrá un catálogo de
	proveedores, a la vez se tendrá uno
	para los clientes, para que, a la vez, se
	organicen los pedidos de forma que
	sepamos que cliente realizo el pedido y
	de qué forma podemos contactarlo.
	
	\item Registro e inicio de sesión para cliente:
	\subitem Para que el cliente pueda ingresar a
	nuestra página web y realizar un
	pedido, tendrá que primero registrarse
	en nuestro sitio web e ingresar la
	información que se le pide (Correo,
	nombres, apellidos, contraseña,
	número telefónico, y su RFC), ya que
	esta información se haya ingresado, el
	perfil se creará y el cliente podrá iniciar
	sesión cuando desee.
	
	\item Modificación y eliminación de
	información del cliente:
	\subitem En la página, el cliente podrá ser capaz
	de alterar su información, como su
	nombre, contraseña, correo, etc. Así
	como también eliminar su cuenta si ya
	no requiere de los servicios que le
	ofrecemos, esto solamente podra pasar
	si el usuario no tiene una compra
	registrada en su cuenta, si es asi,
	tendrá que hablar con el administrador
	con anterioridad.
	
	\item Alta y baja de proveedores:
	\subitem Un administrador será capaz de dar el
	alta, así como baja de sus proveedores
	que ya no estén activos con la empresa.
	
	\item Carrito de compras:
	\subitem El cliente podrá seleccionar sus
	productos que desea comprar en una
	lista previa, con la opción de poder
	borrar algún producto, o proceder al
	pago.
	
	\item Recuperación de Contraseña:
	\subitem Si al cliente se le llegara a olvidar su
	contraseña, tendría la opción de
	recuperarla por medio de su correo.
	
	
\end{itemize}


\section{Diagramas del proyecto}
	
	\begin{figure}[htbp]
		\centering
		\includegraphics[width=1\linewidth]{DER.jpeg}
		\caption{Diagrama Entidad relacion.}
		\label{fig: Diagrama Entidad relacion.}
	\end{figure}
	
	\begin{figure}[htbp]
		\centering
		\includegraphics[width=1\linewidth]{MR.jpeg}
		\caption{Diagrama Modelo relacional.}
		\label{fig: Modelo relacional.}
	\end{figure}
	\newpage

	\begin{figure}[htbp]
		\centering
		\includegraphics[width=1\linewidth]{casosUso.jpeg}
		\caption{Diagrama Casos de uso.}
		\label{fig: Diagrama casos de uso}
	\end{figure}
	
\newpage
	
\section{Mockups}

	\begin{figure}[htbp]
		\centering
		\includegraphics[width=0.5\linewidth]{inicioWeb.jpeg}
		\caption{Página inicial desde la computadora}
		\label{fig: Página inicial desde la computadora.}
	\end{figure}
	
	\begin{figure}[htbp]
		\centering
		\includegraphics[width=0.5\linewidth]{inicioCel.jpeg}
		\caption{Página inicial desde el celular}
		\label{fig: Página inicial desde el celular.}
	\end{figure}
	
	\begin{figure}[htbp]
		\centering
		\includegraphics[width=0.5\linewidth]{catalogoWeb.jpeg}
		\caption{Catálogo de los productos desde la computadora}
		\label{fig: Catálogo de los productos desde la computadora.}
	\end{figure}
	
	\begin{figure}[htbp]
		\centering
		\includegraphics[width=0.5\linewidth]{catalogoCel.jpeg}
		\caption{Catálogo de los productos desde la computadora}
		\label{fig: Catálogo de los productos desde el celular.}
	\end{figure}
	
	\begin{figure}[htbp]
		\centering
		\includegraphics[width=0.5\linewidth]{productoWeb.jpeg}
		\caption{Detalles del producto desde la computadora}
		\label{fig: Detalles del producto desde la computadora.}
	\end{figure}
	
	\begin{figure}[htbp]
		\centering
		\includegraphics[width=0.5\linewidth]{productoCel.jpeg}
		\caption{Detalles del producto desde el celular}
		\label{fig: Detalles del producto desde el celular.}
	\end{figure}

	\begin{figure}[htbp]
		\centering
		\includegraphics[width=0.5\linewidth]{carritoWeb.jpeg}
		\caption{Carrito de compras desde la computadora}
		\label{fig: Carrito de compras desde la computadora.}
	\end{figure}

	\begin{figure}[htbp]
		\centering
		\includegraphics[width=0.5\linewidth]{carritoCel.jpeg}
		\caption{Carrito de compras desde el celular}
		\label{fig: Carrito de compras desde el celular.}
	\end{figure}
	
\end{document}
