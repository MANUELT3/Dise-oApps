\documentclass[journal]{IEEEtran}
\usepackage{apacite}
\bibliographystyle{apacite}

\title{Mobile Devices Architecture}
\author{Jesus Manuel Armenta Telles  \\ \large February 3, 2024}

\begin{document}
	
	\maketitle
	
	\section{Evolution of Mobile Devices Architecture}
	From the early cell phones to today's smartphones, the architecture of mobile devices has undergone several stages of evolution. Initially, cell phones were primarily designed for voice communication, with a simple architecture that included a basic CPU and limited memory. However, with the advancement of technology, mobile devices have evolved to support a wide range of functions, including web browsing, multimedia playback, gaming, and more. This evolution has led to more complex architectures, with multiple CPU cores, integrated GPUs, RAM, and flash storage.
	
	\section{Main Components}
	The architecture of a typical mobile device includes several main components, each of which plays a crucial role in its operation. Some of these components include:
	
	\subsection{Central Processing Unit (CPU)}
	The CPU is the brain of the mobile device, responsible for executing software instructions and performing calculations. In modern mobile devices, CPUs are often multi-core to enhance performance and energy efficiency.
	
	\subsection{Memory}
	Mobile devices use different types of memory to store data and programs. This includes RAM for temporary data storage and flash memory for permanent storage of files and applications.
	
	\subsection{Mobile Operating System}
	The mobile operating system is the software that manages the device's resources and provides an interface for users to interact with it. Some of the most popular mobile operating systems include Android, iOS, and Windows Phone.
	
	\section{Current Challenges}
	Despite advances in mobile device architecture, there are still several significant challenges facing designers and manufacturers. These challenges include:
	
	\subsection{Performance}
	As mobile applications become more sophisticated, there is a growing demand for devices with higher performance and processing capability.
	
	\subsection{Power Consumption}
	Mobile devices must balance performance with power consumption to ensure adequate battery life. This can be challenging, especially in devices with large screens and resource-intensive features.
	
	\section{Future Trends}
	The architecture of mobile devices will continue to evolve in the future to meet changing demands from users and emerging technologies. Some future trends may include:
	
	\subsection{Integrated Artificial Intelligence}
	Mobile devices may increasingly integrate artificial intelligence capabilities to enhance the user experience and support new applications and services.
	
	\subsection{5G Connectivity}
	Widespread adoption of 5G technology will provide faster connection speeds and lower latency, enabling new mobile experiences and services.
	
	\bibliography{references}
	
	
\end{document}
